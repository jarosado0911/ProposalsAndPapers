\label{sec:Project-Details}
The simulations of spatio-temporal dynamics outlined in Sec. \ref{sec:research_plan} will be carried out with the multiphysics simulation package uG4 \citep{Vogel2013, Heppner13}. This is a decades-old project focussed on solving partial differential equations on unstructured grids. In particular the code has been developed to run very well on highly-parallel compute architectures \citep{Vogel2013, Heppner13}. For this proposal we will be making use of \textbf{Finite Element/Finite Volume} discretization methods for the underlying systems of partial differential equations. The resulting system of linear equations will be solved using \textbf{Geometric Multigrid} (GMG) or \textbf{Algebraic Multigrid} (AMG) methods, \textbf{Krylov methods}, or a combination of the above where a Multigrid method may function as a preconditioner. In the subproject \emph{DSMG} we intend to test a novel GMG approach that makes use of \textbf{dimension-switching during coarse grid correction} in a highly parallelized situation. 
The simulation framework uG4 is publicly available on Github\footnote{See the various repositories in the organization \texttt{UG4} on \url{https://github.com/UG4}}. Newly developed code will be tested locally, on local clusters and eventually on remote highly parallel architectures. Upon peer-reviewed publication all code will be made publicly available through a repository in the existing Github organization.
Produced simulation results are stored in VTK format (1 file per time step). After completion of a simulation run, e.g. parameter sweeps, we will copy data to a local storage facility. Over the period of an average month we expect to accumulate $<$ 1 TiB of VTK data. 
