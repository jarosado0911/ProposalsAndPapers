The charge unit for SDSC Comet is the Service Unit (SU). One SU corresponds to the use of one compute
core for one hour, accordingly the requested SUs for the sub-projects in this proposal are summarized
in Tab. \ref{tab:sus} below, which are calculated as: $$SUs = \sum Runtime~[h]\quad \times\quad \#~Cores$$
Since the charge unit for TACC Stampede2 varies between the value of 0.8 SU (KNL queue) and 1.0 SU (SKX queue) we estimate the runtimes on Stampede2 with the greater value to be on the safe side.

\begin{table}
\resizebox{\columnwidth}{!}{%
\tiny
\begin{tabular}{llrrrrr}
   \hline\hline
   \multicolumn{7}{l}{Compute time requirements for \textbf{SDSC Comet}}\\
   \hline
   Sub-project &  Type     & \# runs   & \# steps/ & Wall time/    & \# cores/ & Total \\
            &  of run      &           &  run      &  step [hours] &  run      &  [core-h] (SU) \\
   \hline\hline
   PNP      & production  &   19200    &    10     &   0.01         &  1,200    &  2,304,000 \\
   \hline
   DMSG     & development  &    50     &    10      &     0.3       &  2,000   &   300,000 \\
            & production   &   100     &    10      &     0.1       & 3,200   &   320,000 \\
   \hline
   NET/CD   & development  &    10     &    10     &   0.1         &  1,200    &     9,600 \\
            & production   &  1000     &    10     &   0.01        &  2,400    &   240,000 \\
   \hline
   TOTAL    &              &           &           &               &           &  3,053,600 \\
   \hline\hline
\end{tabular}}
\caption{Total number of SUs on SDSC Comet for the sub-projects requested}
\label{tab:sus}
\end{table}

\begin{table}
\resizebox{\columnwidth}{!}{%
\tiny
\begin{tabular}{llrrrrr}
   \hline\hline
   \multicolumn{7}{l}{Compute time requirements for \textbf{TACC Stampede2}}\\
   \hline
   Sub-project &  Type     & \# runs   & \# steps/ & Wall time/    & \# cores/ & Total \\
            &  of run      &           &  run      &  step [hours] &  run      &  [core-h] (SU) \\
   \hline\hline
   PNP      & production   &    20     &    10     &   0.05        &  64,000    &  640,000 \\
   \hline\hline
   DSMG 
            & production   &    10     &    10     &   0.1        &  64,000    &   640,000  \\
    
   \hline
   TOTAL    &              &           &           &               &           &   1,280,000 \\
   \hline\hline
\end{tabular}}
\caption{Total number of SUs on TACC Stampede2 for the sub-projects requested}
\label{tab:sus2}
\end{table}

During simulations data will be stored in the common VTK format. Since dealing with
reconstructions of spines and a large database sweep of to-be-reconstructed cells
additional mid-term storage on Data Oasis of approximately 700 GiB will be requested.
The amount stems from the following calculation: The average grid size of our model
domain is in the range of about 10-100 MiB and one deals with time-dependent problems,
thus, each timestep will save a snapshot of the simulation. Carrying out multiple
runs for production, e.g. 40, cf. Tab. \ref{tab:sus} PNP problem third column.
and 25 timesteps will already use roughly 100 GiB of data. For DSMG problems grids 
need to be refined but less timesteps are carried out typically, still the degree of
freedoms of the refined grid octuplicate per refinement, thus the production run 
will amount for roughly 100 GiB as well. The NET/CD problems will amount for roughly 
500 GiB as networks of cells tend to be typically larger. In total we request for the development and parameter 
study of the DSMG subproject with application to signal propgation in neurons and networks of neurons 
approximately \textbf{3,053,600 SUs} on SDSC Comet and 700 GiB of medium-term storage on SDSC Data Oasis.
In addition we request in total \textbf{1,280,000} SUs on TACC Stampede2 for further scaling studies
and for longer running simulations of neurobiological problems of interest. Storage will
not exceed 1 TiB of data on the TACC Ranch system, cf. Tab. \ref{tab:sus2}.
