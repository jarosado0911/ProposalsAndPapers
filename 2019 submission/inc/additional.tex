Dr. Queisser (PI) and his research group has long-standing experience in Scientific Computing. Prior to Dr. Queisser's relocation to the US from Germany, his group was awarded resources on the JUQUEEN and JUWELS supercomputers in J\"ulich, Germany over the period of 6 consecutive years. Currently, Dr. Queisser's postdoctoral researcher Dr. Guan and graduate student James Rosado are actively working on the TMS-related subprojects (PNP, DSMG) which are funded through a recently awarded NIH grant. Both have experience in running code on remote parallel compute clusters. Furthermore, graduate student Stephan Grein is involved in the subprojects \emph{DSMG} and \emph{NET/CD} and ran the presented XSEDE-benchmarks. The requested SU are vital for carrying out the NIH project and for the completion of Grein's dissertational research. This project will support 80\% of the National Health Institute (NIH) grant \emph{CRCNS: Collaboration toward an experimentally validated multiscale model of rTMS}. The remaining 20\% are requested to support Grein's dissertation completion. 
Our group has access to the smaller-sized parallel computing system Owlsnest operated by
Temple University, Philadelphia, with roughly 180 general computing CPUs plus some GPU
 capabilities on other nodes. While developing on local multicore machines, then porting to the local compute
cluster Owlsnest for testing, debugging and running smaller batch parameter sweeps is feasible on the local infrastructure, we require access to larger machines for running production code simulations to validate publishable research results. These simulations are beyond what we can run locally under the given constraints. 
