The SDSC Comet CPU resource is suitable for our research because it allows
capacity computing and a quick turnaround on small to medium-scaled jobs
for shorter runtimes, which makes it ideal for running exploratory simulations 
with new code features. The large shared memory per node (128 GiB) allows us to load 
large computational grids which would not be possible with the smaller shared memory
per node on comparable resources. Additionally, the TACC Stampede2 system would 
allow us to carry out capability computations and would
be our primary resource for testing scalability of our applications and performing 
fine grain simulations for scientific publication. Production simulations for
 our neurobiological motivated research questions will potentially have a long 
runtime, and Stampede2 allows for long simulations of about 48 hours in the normal 
queues and one might request even longer running jobs in the long queue.
Comparing our demands with the XSEDE Resource Selector\footnote{Information was 
retrieved from \url{https://portal.xsede.org/allocations/resource-info}}
which facilitates users to make a well-grounded decision on how to select a 
compute resource, confirmed our pre-selected resources. Application efficiency 
is documented in the attached file \textit{performance}.
