For the proposed sub-projects, the authors have established models, numerical methods and tools in the area of detailed 3d simulations of neuronal processes and multiscale modeling. Having developed the tool \emph{NeuRA} for automated reconstruction of the three-dimensional morphologies of neurons and organelles, \cite{Jungblut2011}, and implemented in CUDA for parallel GPU image reconstruction, the detailed morphology of neuronal architectures and organelles can efficiently be incorporated as a parameter for studies with respect to the morphological influence on signal processing in neurons. Numerical simulations, e.g. of the calcium dynamics within nuclei of hippocampal neurons  \cite{Queisser2008, Wittmann2009}, or the vesicular dynamics in boutons of the \emph{Drosophila} Neuromuscular Junction \cite{Knodel2014}, revealed that the three-dimensional organization of the cellular and intracellular domain can significantly influence the functional implications of biochemical cellular signals. 
With the launch of a project between the groups of Andreas Herz (LMU M\"unchen), Gillian Queisser (Goethe University Frankfurt) and Mark Ellisman (UCSD, San Diego, USA) to decipher the dynamical microscale structure-function relation of dendritic spines, a hybrid 1d/3d model of the Poisson-Nernst-Planck equation was implemented in the simulation environment \ug \cite{Vogel2013, Heppner13}. It is intended to identify the functional consequences of the fine-scale organization of dendritic spines on the macroscopic and system level. The complexity of the underlying numerical problem requires high performance computing in order to solve even reduced test cases. 
These aspects are being further investigated in the context of Transcranial Magnetic Stimulation, a clinical technique used to treat neurological diseases, such as schizophrenia, in an NIH-funded project together with Dr. Opitz (Engineering, University of Minnesota, Minneapolis), Dr. Vlachos (Neurophysiology, University of Freiburg, Germany) and Dr. Jedlicka (Computational Neuroscience, University of Giessen, Germany). 
%Preliminary studies of synaptic contacts between nerve cells have revealed, that the positioning of receptors, the position of vesicle release and the size of the synaptic cleft control the structural switch of N-Cadherin trans-cellular connections \cite{Grein2013, Grein2014b}. Building upon this result, a multi-scale model of the synaptic cleft is being developed, including the macroscopic propagation of ions and neurotransmitters (modeled on the continuum scale) and the microscopic molecular dynamics of receptor/neurotransmitter interaction that triggers postsynaptic currents. The full multiscale coupling is achieved by the integration of Molecular Dynamics simulators, such as NAMD, \cite{Humphrey1996, Phillips2005} or MDCore, in the simulation platform \ug, \cite{Heppner13}. Both tools have been shown to carry excellent scaling properties on the BG/Q architecture, \cite{NamdSC12}.
%
%Focussing on the interaction between N-Cadherin and calcium ions, there remains an open question, whether there may be other factors involved in the binding process between the two, rather than the negatively charged Cadherin sidechains and the divalent positive ion. In preliminary studies carried out using the Molecular Dynamics simulator NAMD, \cite{Phillips2005, Humphrey1996}, the authors have investigated the role of dehydration of the binding site as an additional determinant of calcium binding \cite{Grein2014a}.
In addition to the specific neuroscientific applications, there is a vested interest and activity in developing scalable algorithms and methods for efficiently solving numerical problems of the sort mentioned below. This includes robust solvers for highly nonlinear problems such as the Poisson-Nernst-Planck equations or efficient geometry and grid handling. A major aspect of neuroscientific simulation is the underlying computational domain. Nerve cells typically have very anisotropic structures (tree-like shapes), termed dendrites and axons, that elicit complex branching patterns. A numerically robust discrete representation of these morphologies in the form of volume meshes is needed, that can undergo stable grid-refinement within multi-level solvers, such as geometric multigrid methods. The applicants have developed a novel grid refinement strategy, based on the subdivision volume theory, that produces ideal mixed-element volume grid refinements which can handle highly complex biological morphologies. Preliminary tests show this robustness. More extensive studies in application-relevant benchmarks again require ample computing resources.
