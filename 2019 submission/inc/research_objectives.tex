Computational Neuroscience is a rapidly developing research field that demands more and more computing resources for solving complex problems. Where in the past, models of signal processing in brain cells were strongly reduced -- due to the lack of computing power and high resolution experimental data -- they are now being developed in more detail. Emerging areas of research encompass large scale network simulations, three-dimensional simulations of neuronal signals \cite{Xylouris2007, Xylouris2012, Breit2016, Breit2018a, Breit2018b} and multiscale modeling and simulation in neuroscience \cite{Stepniewski2019b}.

In order to resolve biochemical and electrical signals of neurons and networks based on physical first principles, models defined by sets of highly nonlinear and coupled partial and ordinary differential equations need to be numerically and efficiently solved. Models based on partial differential equations (PDEs) furthermore need to be coupled to microscale processes, such as the molecular dynamics of post-synaptic receptors. Thus, solvers for numerical PDEs and molecular dynamics need to be coupled in a multiscale modeling and simulation approach. Resolving ion dynamics at high intracellular resolution is, even with ample computing resources, challenging. In order to cope with the computational complexity of these problems, hybrid dimensional methods need to be developed, where e.g. \emph{high interest} zones are resolved in full three dimensions, whereas \emph{low interest} zones are resolved in two or one dimensions.

The main project below covers the areas of highly detailed three-dimensional modeling and simulation of neuronal processes, multiscale modeling as well as the development of numerical methods for efficiently solving the arising numerical problems on highly parallel computing platforms. The intention within this project is to develop novel numerical strategies and to compute large sweeps of neuroscientific simulation data, in order to support and advance active projects in Computational Neuroscience.
